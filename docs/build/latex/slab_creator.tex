%% Generated by Sphinx.
\def\sphinxdocclass{report}
\documentclass[letterpaper,10pt,english]{sphinxmanual}
\ifdefined\pdfpxdimen
   \let\sphinxpxdimen\pdfpxdimen\else\newdimen\sphinxpxdimen
\fi \sphinxpxdimen=.75bp\relax
\ifdefined\pdfimageresolution
    \pdfimageresolution= \numexpr \dimexpr1in\relax/\sphinxpxdimen\relax
\fi
%% let collapsible pdf bookmarks panel have high depth per default
\PassOptionsToPackage{bookmarksdepth=5}{hyperref}

\PassOptionsToPackage{booktabs}{sphinx}
\PassOptionsToPackage{colorrows}{sphinx}

\PassOptionsToPackage{warn}{textcomp}
\usepackage[utf8]{inputenc}
\ifdefined\DeclareUnicodeCharacter
% support both utf8 and utf8x syntaxes
  \ifdefined\DeclareUnicodeCharacterAsOptional
    \def\sphinxDUC#1{\DeclareUnicodeCharacter{"#1}}
  \else
    \let\sphinxDUC\DeclareUnicodeCharacter
  \fi
  \sphinxDUC{00A0}{\nobreakspace}
  \sphinxDUC{2500}{\sphinxunichar{2500}}
  \sphinxDUC{2502}{\sphinxunichar{2502}}
  \sphinxDUC{2514}{\sphinxunichar{2514}}
  \sphinxDUC{251C}{\sphinxunichar{251C}}
  \sphinxDUC{2572}{\textbackslash}
\fi
\usepackage{cmap}
\usepackage[T1]{fontenc}
\usepackage{amsmath,amssymb,amstext}
\usepackage{babel}



\usepackage{tgtermes}
\usepackage{tgheros}
\renewcommand{\ttdefault}{txtt}



\usepackage[Bjarne]{fncychap}
\usepackage{sphinx}

\fvset{fontsize=auto}
\usepackage{geometry}


% Include hyperref last.
\usepackage{hyperref}
% Fix anchor placement for figures with captions.
\usepackage{hypcap}% it must be loaded after hyperref.
% Set up styles of URL: it should be placed after hyperref.
\urlstyle{same}

\addto\captionsenglish{\renewcommand{\contentsname}{Contents:}}

\usepackage{sphinxmessages}
\setcounter{tocdepth}{1}



\title{slab\_creator}
\date{Mar 28, 2024}
\release{v1}
\author{Kazem Zhour}
\newcommand{\sphinxlogo}{\vbox{}}
\renewcommand{\releasename}{Release}
\makeindex
\begin{document}

\ifdefined\shorthandoff
  \ifnum\catcode`\=\string=\active\shorthandoff{=}\fi
  \ifnum\catcode`\"=\active\shorthandoff{"}\fi
\fi

\pagestyle{empty}
\sphinxmaketitle
\pagestyle{plain}
\sphinxtableofcontents
\pagestyle{normal}
\phantomsection\label{\detokenize{index::doc}}


\sphinxAtStartPar
The Slab Creator script (slab\_creator.py) facilitates the creation of slab structures with different orientations.

\sphinxstepscope


\chapter{Usage}
\label{\detokenize{usage:usage}}\label{\detokenize{usage::doc}}
\sphinxAtStartPar
To use the script, follow these steps:
\begin{enumerate}
\sphinxsetlistlabels{\arabic}{enumi}{enumii}{}{.}%
\item {} 
\sphinxAtStartPar
Launch the script by running the slab\_creator.py file.

\item {} 
\sphinxAtStartPar
Provide the path to the CONTCAR file containing the bulk structure.

\item {} 
\sphinxAtStartPar
Enter the Miller indices for the desired slab orientation.

\item {} 
\sphinxAtStartPar
Specify the minimum slab size and minimum vacuum size in Angstroms.

\item {} 
\sphinxAtStartPar
Choose whether to apply LLL reduction (orthogonalize) to the slab.

\item {} 
\sphinxAtStartPar
Choose whether to force the slab to be perpendicular to the c\sphinxhyphen{}axis.

\end{enumerate}

\sphinxAtStartPar
The generated slab will be saved in VASP format with the specified Miller indices in the filename (e.g., POSCAR100.vasp). Additionally, the script will visualize the slab structure using the ASE package in the gui version of the code.

\sphinxstepscope


\chapter{Examples}
\label{\detokenize{examples:examples}}\label{\detokenize{examples::doc}}
\sphinxAtStartPar
Here’s an example of how to run the script:

\sphinxAtStartPar
python slab\_creator.py

\sphinxAtStartPar
Follow the on\sphinxhyphen{}screen instructions to provide the required inputs and generate the slab structure.

\sphinxAtStartPar
Or you can run the GUI version of the code in the gui directory:

\sphinxAtStartPar
python slab\_creator+gui.py

\sphinxAtStartPar
then fill the required variables.

\sphinxstepscope


\chapter{Dependencies}
\label{\detokenize{dependencies:dependencies}}\label{\detokenize{dependencies::doc}}
\sphinxAtStartPar
The Slab Creator (slab\_creator.py) script requires the following dependencies:
\begin{itemize}
\item {} 
\sphinxAtStartPar
pymatgen: A Python library for materials analysis.

\end{itemize}

\sphinxAtStartPar
pip install pymatgen
\begin{itemize}
\item {} 
\sphinxAtStartPar
ase: Another Python library for materials analysis (required only for the GUI version of the script).

\end{itemize}

\sphinxAtStartPar
pip install ase
\begin{itemize}
\item {} 
\sphinxAtStartPar
tkinter: The standard Python interface to the Tk GUI toolkit (required only for the GUI version of the script).

\end{itemize}


\chapter{Indices and tables}
\label{\detokenize{index:indices-and-tables}}\begin{itemize}
\item {} 
\sphinxAtStartPar
\DUrole{xref,std,std-ref}{genindex}

\item {} 
\sphinxAtStartPar
\DUrole{xref,std,std-ref}{modindex}

\item {} 
\sphinxAtStartPar
\DUrole{xref,std,std-ref}{search}

\end{itemize}



\renewcommand{\indexname}{Index}
\printindex
\end{document}